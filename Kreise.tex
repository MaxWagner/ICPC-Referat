\section{Kreise}
\begin{frame}
	\tableofcontents[ 
	sectionstyle=show/hide
	]
\end{frame}

\subsection{Eulerkreisproblem}

\begin{frame}{Eulerkreisproblem}{Eulerian path}
	\begin{block}{Eulerkreis}
		Ein Eulerkreis in einem Graphen $(V,E)$ ist ein Tupel $(e_1,e_2,...,e_n)$ mit $e_i \in E$ für $i = 1,...,n$ und $\left|E\right| = n$. Hierbei soll gelten: $i \neq j \Rightarrow e_i \neq e_j$. Des weiteren sollen für $j = i+1$ $e_i$ und $e_j$ jeweils zum selben Knoten inzident sein, und $e_1$ soll zum gleichen Knoten inzident sein wie $e_n$
	\end{block}

	\begin{block}{Erklärung}
		Ein Eulerkreis ist ein Zirkel in einem Graphen, in dem jede Kante genau einmal benutzt wird.
	\end{block}
\end{frame}

\begin{frame}
	\begin{figure}
		\begin{tikzpicture}[->,scale=1.8, auto,swap]
			% Draw a 7,11 network
			% First we draw the vertices
			\foreach \pos/\name in {{(0,0)/a}, {(0,2)/b}, {(1,2)/c},
				                    {(1,0)/d}, {(2,1)/e}, {(3,1)/f}, 
									{(4,2)/g}, {(5,2)/h}, {(4,0)/i},
									{(5,0)/j}}
				\node[vertex] (\name) at \pos {$\name$};
			% Connect vertices with edges and draw weights
			\foreach \source/ \dest /\pos in {a/b/,b/c/,c/d/,d/a/,
										c/e/bend left, d/e/,e/c/,
										e/f/,
										f/g/, f/i/,g/f/bend right,i/f/bend left,
										g/h/, h/j/, j/i/, i/g/}
				\path (\source) edge [\pos] node {} (\dest);
		\end{tikzpicture}
	\end{figure}
\end{frame}

\subsection{Hamiltonkreisproblem}
\begin{frame}{Hamiltonkreisproblem}{Hamiltonian path}
	\begin{block}{Hamiltonkreis}
		Ein Hamiltonkreis in einem Graphen $(V,E)$ ist ein Tupel $(v_1,v_2,...,v_n)$ mit $v_i \in V$ für $i = 1,...,n$ und $\left|V\right| = n$. Hierbei soll gelten: $i \neq j \Rightarrow v_i \neq v_j$. Des weiteren sollen für $j = i+1$ $v_i$ und $v_j$ jeweils zueinander adjazent sein, und $v_1$ soll soll adjazent sein zu $v_n$
	\end{block}

	\begin{block}{Erklärung}
		Ein Hamiltonkreis ist ein Zirkel in einem Graphen, in dem jeder Knoten genau einmal benutzt wird.
	\end{block}
\end{frame}

\begin{frame}
	\begin{figure}
		\begin{tikzpicture}[->,scale=1.8, auto,swap]
			% Draw a 7,11 network
			% First we draw the vertices
			\foreach \pos/\name in {{(0,0)/a}, {(0,2)/b}, {(1,2)/c},
				                    {(1,0)/d}, {(2,1)/e}, {(3,1)/f}, 
									{(4,2)/g}, {(5,2)/h}, {(4,0)/i},
									{(5,0)/j}}
				\node[vertex] (\name) at \pos {$\name$};
			% Connect vertices with edges and draw weights
			\foreach \source/ \dest /\pos in {a/b/,b/c/,c/d/,d/a/,
										c/e/bend left, d/e/,e/c/,
										e/f/,
										f/g/, f/i/,g/f/bend right,i/f/bend left,
										g/h/, h/j/, j/i/, i/g/}
				\path (\source) edge [\pos] node {} (\dest);
		\end{tikzpicture}
	\end{figure}
\end{frame}
