\documentclass[hyperref={pdfpagelabels=false}]{beamer}
\usepackage{lmodern}

\usepackage[utf8]{inputenc} % this is needed for german umlauts
\usepackage[ngerman]{babel} % this is needed for german umlauts
\usepackage[T1]{fontenc}    % this is needed for correct output of umlauts in pdf

\usepackage{verbatim}
\usepackage{tikz}
\usetikzlibrary{arrows,shapes}

% Define some styles for graphs
\tikzstyle{vertex}=[circle,fill=black!25,minimum size=20pt,inner sep=0pt]
\tikzstyle{selected vertex} = [vertex, fill=red!24]
\tikzstyle{edge} = [draw,thick,-]
\tikzstyle{weight} = [font=\small]
\tikzstyle{selected edge} = [draw,line width=5pt,-,red!50]
\tikzstyle{ignored edge} = [draw,line width=5pt,-,black!20]

\usetheme{Frankfurt} % see http://deic.uab.es/~iblanes/beamer_gallery/index_by_theme.html
\usefonttheme{professionalfonts}
\beamertemplatenavigationsymbolsempty

\begin{document}
\title{Graphentheorie II}   
\author{Martin Thoma, Tobias Sturm} 
\date{\today} 
\subject{Graphentheorie-Referat fur ICPC}

\frame{\titlepage} 

\frame{
	\frametitle{Inhaltsverzeichnis}
	\tableofcontents[section]
}

\section{Minimale Spannbäume}
\begin{frame}{Minimale Spannbäume}{Minimal Spanning Trees}
	\begin{itemize}
		\item Algorithmus von Kruskal
		\item Algorithmus von Prim
	\end{itemize}
\end{frame}

% Author: Kjell Magne Fauske
% Source: http://www.texample.net/tikz/examples/prims-algorithm/
% Declare layers
\pgfdeclarelayer{background}
\pgfsetlayers{background,main}

\subsection{Algorithmus von Prim}
\begin{frame}{Algorithmus von Prim}{Prim's algorithm}
	%% Adjacency matrix of graph
	%% \  a  b  c  d  e  f  g
	%% a  x  7     5
	%% b  7  x  8  9  7
	%% c     8  x     5
	%% d  5  9     x 15  6
	%% e     7  5 15  x  8  9
	%% f           6  8  x 11
	%% g              9  11 x
	\begin{figure}
		\begin{tikzpicture}[scale=1.8, auto,swap]
			% Draw a 7,11 network
			% First we draw the vertices
			\foreach \pos/\name in {{(0,2)/a}, {(2,1)/b}, {(4,1)/c},
				                    {(0,0)/d}, {(3,0)/e}, {(2,-1)/f}, {(4,-1)/g}}
				\node[vertex] (\name) at \pos {$\name$};
			% Connect vertices with edges and draw weights
			\foreach \source/ \dest /\weight in {b/a/7, c/b/8,d/a/5,d/b/9,
				                                 e/b/7, e/c/5,e/d/15,
				                                 f/d/6,f/e/8,
				                                 g/e/9,g/f/11}
				\path[edge] (\source) -- node[weight] {$\weight$} (\dest);
			% Start animating the vertex and edge selection. 
			\foreach \vertex / \fr in {d/1,a/2,f/3,b/4,e/5,c/6,g/7}
				\path<\fr-> node[selected vertex] at (\vertex) {$\vertex$};
			% For convenience we use a background layer to highlight edges
			% This way we don't have to worry about the highlighting covering
			% weight labels. 
			\begin{pgfonlayer}{background}
				\pause
				\foreach \source / \dest in {d/a,d/f,a/b,b/e,e/c,e/g}
				    \path<+->[selected edge] (\source.center) -- (\dest.center);
				\foreach \source / \dest / \fr in {d/b/4,d/e/5,e/f/5,b/c/6,f/g/7}
				    \path<\fr->[ignored edge] (\source.center) -- (\dest.center);
			\end{pgfonlayer}
		\end{tikzpicture}
	\end{figure}
\end{frame}

% Author: Martin Thoma
\subsection{Algorithmus von Kruskal}
\begin{frame}{Kruskal's algorithm}
	\begin{figure}
		\begin{tikzpicture}[scale=1.8, auto,swap]
			% Draw a 7,11 network
			% First we draw the vertices
			\foreach \pos/\name in {{(0,2)/a}, {(2,1)/b}, {(4,1)/c},
				                    {(0,0)/d}, {(3,0)/e}, {(2,-1)/f}, {(4,-1)/g}}
				\node[vertex] (\name) at \pos {$\name$};
			% Connect vertices with edges and draw weights
			\foreach \source/ \dest /\weight in {b/a/7, c/b/8,d/a/5,d/b/9,
				                                 e/b/7, e/c/5,e/d/15,
				                                 f/d/6,f/e/8,
				                                 g/e/9,g/f/11}
				\path[edge] (\source) -- node[weight] {$\weight$} (\dest);
			% Start animating the vertex and edge selection. 
			\foreach \vertex / \fr in {d/1,a/1,e/2,c/2,f/3,b/4,g/10}
				\path<\fr-> node[selected vertex] at (\vertex) {$\vertex$};
			% For convenience we use a background layer to highlight edges
			% This way we don't have to worry about the highlighting covering
			% weight labels. 
			\begin{pgfonlayer}{background}
				\pause
				\foreach \source / \dest / \fr in {a/d/1,c/e/2,d/f/3,a/b/4,b/e/6,e/g/10}
				    \path<\fr->[selected edge] (\source.center) -- (\dest.center);
				\foreach \source / \dest / \fr in {d/b/5,b/c/7,d/e/8,e/f/9,f/g/11}
				    \path<\fr->[ignored edge] (\source.center) -- (\dest.center);
			\end{pgfonlayer}
		\end{tikzpicture}
	\end{figure}
\end{frame}
       % Minimale Spannbäume
\section{Starke Zusammenhangskomponenten}

\subsection*{Starke Zusammenhangskomponenten}
\begin{frame}{Starke Zusammenhangskomponenten}{Strongly connected components}
	\begin{block}{Starke Zusammenhangskomponente}
		Ein induzierter Teilgraph $G[U]$ für eine Teilmenge 
		$U \subset V$ heißt starke Zusammenhangskomponente von $G$, 
		falls $G[U]$ stark zusammenhängend ist und kein stark 
		zusammenhängender induzierter Teilgraph von $G$ existiert, 
		der $G[U]$ echt enthält.
	\end{block}
\end{frame}

\subsection*{Worum geht es?}
\begin{frame}{Worum geht es?}{}
	Gegeben ist ein Graph G(V, E):

	\begin{figure}
		\begin{tikzpicture}[->,scale=1.8, auto,swap]
			% Draw a 7,11 network
			% First we draw the vertices
			\foreach \pos/\name in {{(0,0)/a}, {(0,2)/b}, {(1,2)/c},
				                    {(1,0)/d}, {(2,1)/e}, {(3,1)/f}, 
									{(4,2)/g}, {(5,2)/h}, {(4,0)/i},
									{(5,0)/j}}
				\node[vertex] (\name) at \pos {$\name$};
			% Connect vertices with edges and draw weights
			\foreach \source/ \dest /\pos in {a/b/,b/c/,c/d/,d/a/,
										c/e/bend left, d/e/,e/c/,
										e/f/,
										f/g/, f/i/,g/f/bend right,i/f/bend left,
										g/h/, h/j/, j/i/, i/g/}
				\path (\source) edge [\pos] node {} (\dest);
		\end{tikzpicture}
	\end{figure}

	Frage: Gibt es Teilgraphen G'(V', E') mit $V' \subset V$ und $E' \subset E$, sodass gilt:
	$\forall a, b \in V: \exists \text{Pfad von a nach b} \in G$
\end{frame}

\begin{frame}{Worum geht es?}{}
	Antwort: Ja, gibt es:

	\begin{figure}
		\begin{tikzpicture}[->,scale=1.8, auto,swap]
			% Draw a 7,11 network
			% First we draw the vertices
			\foreach \pos/\name in {{(0,0)/a}, {(0,2)/b}, {(1,2)/c},
				                    {(1,0)/d}, {(2,1)/e}, {(3,1)/f}, 
									{(4,2)/g}, {(5,2)/h}, {(4,0)/i},
									{(5,0)/j}}
				\node[vertex] (\name) at \pos {$\name$};
			% Connect vertices with edges
			\foreach \source/ \dest /\pos in {a/b/,b/c/,c/d/,d/a/,
										c/e/bend left, d/e/,e/c/,
										e/f/,
										f/g/, f/i/,g/f/bend right,i/f/bend left,
										g/h/, h/j/, j/i/, i/g/}
				\path (\source) edge [\pos] node {} (\dest);
			\pause \path<+-> node[selected vertex] at (a) {$a$};
			\begin{pgfonlayer}{background} \path<+->[selected edge] (a.center) -- (b.center); \end{pgfonlayer}
			\path<+-> node[selected vertex] at (b) {$b$};
			\begin{pgfonlayer}{background} \path<+->[selected edge] (b.center) -- (c.center); \end{pgfonlayer}
			\path<+-> node[selected vertex] at (c) {$c$};
			\begin{pgfonlayer}{background} \path<+->[selected edge] (c.center) -- (d.center); \end{pgfonlayer}
			\path<+-> node[selected vertex] at (d) {$d$};
			\begin{pgfonlayer}{background} \path<+->[selected edge] (d.center) -- (a.center); \end{pgfonlayer}
		\end{tikzpicture}
	\end{figure}
\end{frame}

\subsection*{Brücke}
\begin{frame}{Brücke}{Bridge}
	\begin{figure}
		\begin{tikzpicture}[->,scale=1.8, auto,swap]
			% Draw a 7,11 network
			% First we draw the vertices
			\foreach \pos/\name in {{(0,0)/a}, {(0,2)/b}, {(1,2)/c},
				                    {(1,0)/d}, {(2,1)/e}, {(3,1)/f}, 
									{(4,2)/g}, {(5,2)/h}, {(4,0)/i},
									{(5,0)/j}}
				\node[vertex] (\name) at \pos {$\name$};
			% Connect vertices with edges
			\foreach \source/ \dest /\pos in {a/b/,b/c/,c/d/,d/a/,
										c/e/bend left, d/e/,e/c/,
										e/f/,
										f/g/, f/i/,g/f/bend right,i/f/bend left,
										g/h/, h/j/, j/i/, i/g/}
				\path (\source) edge [\pos] node {} (\dest);
			\begin{pgfonlayer}{background} \path<+->[selected edge] (e.center) -- (f.center); \end{pgfonlayer}
		\end{tikzpicture}
	\end{figure}
\end{frame}

\subsection*{Artikulationspunkt}
\begin{frame}{Artikulationspunkt}{Articulation vertex or cut vertices}
	Auch "Gelenkpunkt" genannt
\end{frame}

\subsection*{Zweifachverbundener Graph}
\begin{frame}{Zweifachverbundener Graph}{Biconnected graph}
	\begin{figure}
		\begin{tikzpicture}[->,scale=1.8, auto,swap]
			% Draw a 7,11 network
			% First we draw the vertices
			\foreach \pos/\name in {{(0,0)/a}, {(0,2)/b}, {(1,2)/c},
				                    {(1,0)/d}, {(2,1)/e}, {(3,1)/f}, 
									{(4,2)/g}, {(5,2)/h}, {(4,0)/i},
									{(5,0)/j}}
				\node[vertex] (\name) at \pos {$\name$};
			% Connect vertices with edges
			\foreach \source/ \dest /\pos in {a/b/,b/c/,c/d/,d/a/,
										c/e/bend left, d/e/,e/c/,
										e/f/,
										f/g/, f/i/,g/f/bend right,i/f/bend left,
										g/h/, h/j/, j/i/, i/g/}
				\path (\source) edge [\pos] node {} (\dest);
			\path<+-> node[selected vertex] at (e) {$d$};
		\end{tikzpicture}
	\end{figure}
\end{frame}

\subsection*{Tiefensuche}
\begin{frame}{Tiefensuche}{Tiefensuche}
	\begin{block}{Tiefensuche}
		...
	\end{block}
\end{frame}
       % Starke zusammenhangskomponenten
\section{Färbung von Graphen}
\begin{frame}{Färbung von Graphen}{Graph coloring}
	\begin{itemize}
		\item Ist für 2 entscheidbar
		\item Für 3 schon schwer
		\item blub
	\end{itemize}
\end{frame}
  % Färbung von Graphen
\section{Kreise}
\begin{frame}
	\tableofcontents[ 
	sectionstyle=show/hide
	]
\end{frame}

\subsection{Eulerkreisproblem}
\begin{frame}{Eulerkreisproblem}{Eulerian path}
	\begin{figure}
		\begin{tikzpicture}[->,scale=1.8, auto,swap]
			% Draw a 7,11 network
			% First we draw the vertices
			\foreach \pos/\name in {{(0,0)/a}, {(0,2)/b}, {(1,2)/c},
				                    {(1,0)/d}, {(2,1)/e}, {(3,1)/f}, 
									{(4,2)/g}, {(5,2)/h}, {(4,0)/i},
									{(5,0)/j}}
				\node[vertex] (\name) at \pos {$\name$};
			% Connect vertices with edges and draw weights
			\foreach \source/ \dest /\pos in {a/b/,b/c/,c/d/,d/a/,
										c/e/bend left, d/e/,e/c/,
										e/f/,
										f/g/, f/i/,g/f/bend right,i/f/bend left,
										g/h/, h/j/, j/i/, i/g/}
				\path (\source) edge [\pos] node {} (\dest);
		\end{tikzpicture}
	\end{figure}
\end{frame}

\subsection{Hamiltonkreisproblem}
\begin{frame}{Hamiltonkreisproblem}{Hamiltonian path}
	\begin{figure}
		\begin{tikzpicture}[->,scale=1.8, auto,swap]
			% Draw a 7,11 network
			% First we draw the vertices
			\foreach \pos/\name in {{(0,0)/a}, {(0,2)/b}, {(1,2)/c},
				                    {(1,0)/d}, {(2,1)/e}, {(3,1)/f}, 
									{(4,2)/g}, {(5,2)/h}, {(4,0)/i},
									{(5,0)/j}}
				\node[vertex] (\name) at \pos {$\name$};
			% Connect vertices with edges and draw weights
			\foreach \source/ \dest /\pos in {a/b/,b/c/,c/d/,d/a/,
										c/e/bend left, d/e/,e/c/,
										e/f/,
										f/g/, f/i/,g/f/bend right,i/f/bend left,
										g/h/, h/j/, j/i/, i/g/}
				\path (\source) edge [\pos] node {} (\dest);
		\end{tikzpicture}
	\end{figure}
\end{frame}
  % Färbung von Graphen

\end{document}

