\section{Starke Zusammenhangskomponenten}
\begin{frame}
	\tableofcontents[ 
	sectionstyle=show/hide
	]
\end{frame}

\subsection{Starke Zusammenhangskomponenten}
\begin{frame}{Starke Zusammenhangskomponenten}{Strongly connected components}
	\begin{block}{Starke Zusammenhangskomponente}
		Ein induzierter Teilgraph $G[U]$ für eine Teilmenge 
		$U \subset V$ heißt starke Zusammenhangskomponente von $G$, 
		falls $G[U]$ stark zusammenhängend ist und kein stark 
		zusammenhängender induzierter Teilgraph von $G$ existiert, 
		der $G[U]$ echt enthält.
	\end{block}
\end{frame}

\subsection{Worum geht es?}
\begin{frame}{Worum geht es?}{}
	Gegeben ist ein Graph G(V, E):

	\begin{figure}
		\begin{tikzpicture}[->,scale=1.8, auto,swap]
			% Draw a 7,11 network
			% First we draw the vertices
			\foreach \pos/\name in {{(0,0)/a}, {(0,2)/b}, {(1,2)/c},
				                    {(1,0)/d}, {(2,1)/e}, {(3,1)/f}, 
									{(4,2)/g}, {(5,2)/h}, {(4,0)/i},
									{(5,0)/j}}
				\node[vertex] (\name) at \pos {$\name$};
			% Connect vertices with edges and draw weights
			\foreach \source/ \dest /\pos in {a/b/,b/c/,c/d/,d/a/,
										c/e/bend left, d/e/,e/c/,
										e/f/,
										f/g/, f/i/,g/f/bend right,i/f/bend left,
										g/h/, h/j/, j/i/, i/g/}
				\path (\source) edge [\pos] node {} (\dest);
		\end{tikzpicture}
	\end{figure}

	Frage: Gibt es Teilgraphen G'(V', E') mit $V' \subset V$ und $E' \subset E$, sodass gilt:
	$\forall a, b \in V: \exists \text{Pfad von a nach b} \in G$
\end{frame}

\begin{frame}{Worum geht es?}{}
	Antwort: Ja, gibt es:

	\begin{figure}
		\begin{tikzpicture}[->,scale=1.8, auto,swap]
			% Draw a 7,11 network
			% First we draw the vertices
			\foreach \pos/\name in {{(0,0)/a}, {(0,2)/b}, {(1,2)/c},
				                    {(1,0)/d}, {(2,1)/e}, {(3,1)/f}, 
									{(4,2)/g}, {(5,2)/h}, {(4,0)/i},
									{(5,0)/j}}
				\node[vertex] (\name) at \pos {$\name$};
			% Connect vertices with edges
			\foreach \source/ \dest /\pos in {a/b/,b/c/,c/d/,d/a/,
										c/e/bend left, d/e/,e/c/,
										e/f/,
										f/g/, f/i/,g/f/bend right,i/f/bend left,
										g/h/, h/j/, j/i/, i/g/}
				\path (\source) edge [\pos] node {} (\dest);
			\pause \path<+-> node[selected vertex] at (a) {$a$};
			\begin{pgfonlayer}{background} \path<+->[selected edge] (a.center) -- (b.center); \end{pgfonlayer}
			\path<+-> node[selected vertex] at (b) {$b$};
			\begin{pgfonlayer}{background} \path<+->[selected edge] (b.center) -- (c.center); \end{pgfonlayer}
			\path<+-> node[selected vertex] at (c) {$c$};
			\begin{pgfonlayer}{background} \path<+->[selected edge] (c.center) -- (d.center); \end{pgfonlayer}
			\path<+-> node[selected vertex] at (d) {$d$};
			\begin{pgfonlayer}{background} \path<+->[selected edge] (d.center) -- (a.center); \end{pgfonlayer}
		\end{tikzpicture}
	\end{figure}
\end{frame}

\subsection{Brücke}
\begin{frame}{Brücke}{Bridge}
	\begin{figure}
		\begin{tikzpicture}[->,scale=1.8, auto,swap]
			% Draw a 7,11 network
			% First we draw the vertices
			\foreach \pos/\name in {{(0,0)/a}, {(0,2)/b}, {(1,2)/c},
				                    {(1,0)/d}, {(2,1)/e}, {(3,1)/f}, 
									{(4,2)/g}, {(5,2)/h}, {(4,0)/i},
									{(5,0)/j}}
				\node[vertex] (\name) at \pos {$\name$};
			% Connect vertices with edges
			\foreach \source/ \dest /\pos in {a/b/,b/c/,c/d/,d/a/,
										c/e/bend left, d/e/,e/c/,
										e/f/,
										f/g/, f/i/,g/f/bend right,i/f/bend left,
										g/h/, h/j/, j/i/, i/g/}
				\path (\source) edge [\pos] node {} (\dest);
			\begin{pgfonlayer}{background} \path<+->[selected edge] (e.center) -- (f.center); \end{pgfonlayer}
		\end{tikzpicture}
	\end{figure}
\end{frame}

\subsection{Artikulationspunkt}
\begin{frame}{Artikulationspunkt}{Articulation vertex or cut vertices}
	Auch "Gelenkpunkt" genannt
\end{frame}

\subsection{Zweifachverbundener Graph}
\begin{frame}{Zweifachverbundener Graph}{Biconnected graph}
	\begin{figure}
		\begin{tikzpicture}[->,scale=1.8, auto,swap]
			% Draw a 7,11 network
			% First we draw the vertices
			\foreach \pos/\name in {{(0,0)/a}, {(0,2)/b}, {(1,2)/c},
				                    {(1,0)/d}, {(2,1)/e}, {(3,1)/f}, 
									{(4,2)/g}, {(5,2)/h}, {(4,0)/i},
									{(5,0)/j}}
				\node[vertex] (\name) at \pos {$\name$};
			% Connect vertices with edges
			\foreach \source/ \dest /\pos in {a/b/,b/c/,c/d/,d/a/,
										c/e/bend left, d/e/,e/c/,
										e/f/,
										f/g/, f/i/,g/f/bend right,i/f/bend left,
										g/h/, h/j/, j/i/, i/g/}
				\path (\source) edge [\pos] node {} (\dest);
			\path<+-> node[selected vertex] at (e) {$d$};
		\end{tikzpicture}
	\end{figure}
\end{frame}

\subsection{Tiefensuche}
\begin{frame}{Tiefensuche}{Tiefensuche}
	\begin{block}{Tiefensuche}
		...
	\end{block}
\end{frame}
